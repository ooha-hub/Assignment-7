\documentclass[journal,12pt,twocolumn]{IEEEtran}

\usepackage{setspace}
\usepackage{gensymb}

\singlespacing


\usepackage[cmex10]{amsmath}

\usepackage{amsthm}

\usepackage{mathrsfs}
\usepackage{txfonts}
\usepackage{stfloats}
\usepackage{bm}
\usepackage{cite}
\usepackage{cases}
\usepackage{subfig}

\usepackage{longtable}
\usepackage{multirow}

\usepackage{enumitem}
\usepackage{mathtools}
\usepackage{steinmetz}
\usepackage{tikz}
\usepackage{circuitikz}
\usepackage{verbatim}
\usepackage{tfrupee}
\usepackage[breaklinks=true]{hyperref}
\usepackage{graphicx}
\usepackage{tkz-euclide}

\usetikzlibrary{calc,math}
\usepackage{listings}
    \usepackage{color}                                            %%
    \usepackage{array}                                            %%
    \usepackage{longtable}                                        %%
    \usepackage{calc}                                             %%
    \usepackage{multirow}                                         %%
    \usepackage{hhline}                                           %%
    \usepackage{ifthen}                                           %%
    \usepackage{lscape}     
\usepackage{multicol}
\usepackage{chngcntr}

\DeclareMathOperator*{\Res}{Res}

\renewcommand\thesection{\arabic{section}}
\renewcommand\thesubsection{\thesection.\arabic{subsection}}
\renewcommand\thesubsubsection{\thesubsection.\arabic{subsubsection}}

\renewcommand\thesectiondis{\arabic{section}}
\renewcommand\thesubsectiondis{\thesectiondis.\arabic{subsection}}
\renewcommand\thesubsubsectiondis{\thesubsectiondis.\arabic{subsubsection}}


\hyphenation{op-tical net-works semi-conduc-tor}
\def\inputGnumericTable{}                                 %%

\lstset{
%language=C,
frame=single, 
breaklines=true,
columns=fullflexible
}
\begin{document}


\newtheorem{theorem}{Theorem}[section]
\newtheorem{problem}{Problem}
\newtheorem{proposition}{Proposition}[section]
\newtheorem{lemma}{Lemma}[section]
\newtheorem{corollary}[theorem]{Corollary}
\newtheorem{example}{Example}[section]
\newtheorem{definition}[problem]{Definition}

\newcommand{\BEQA}{\begin{eqnarray}}
\newcommand{\EEQA}{\end{eqnarray}}
\newcommand{\define}{\stackrel{\triangle}{=}}
\bibliographystyle{IEEEtran}
\providecommand{\mbf}{\mathbf}
\providecommand{\pr}[1]{\ensuremath{\Pr\left(#1\right)}}
\providecommand{\qfunc}[1]{\ensuremath{Q\left(#1\right)}}
\providecommand{\sbrak}[1]{\ensuremath{{}\left[#1\right]}}
\providecommand{\lsbrak}[1]{\ensuremath{{}\left[#1\right.}}
\providecommand{\rsbrak}[1]{\ensuremath{{}\left.#1\right]}}
\providecommand{\brak}[1]{\ensuremath{\left(#1\right)}}
\providecommand{\lbrak}[1]{\ensuremath{\left(#1\right.}}
\providecommand{\rbrak}[1]{\ensuremath{\left.#1\right)}}
\providecommand{\cbrak}[1]{\ensuremath{\left\{#1\right\}}}
\providecommand{\lcbrak}[1]{\ensuremath{\left\{#1\right.}}
\providecommand{\rcbrak}[1]{\ensuremath{\left.#1\right\}}}
\theoremstyle{remark}
\newtheorem{rem}{Remark}
\newcommand{\sgn}{\mathop{\mathrm{sgn}}}
\providecommand{\abs}[1]{\left\vert#1\right\vert}
\providecommand{\res}[1]{\Res\displaylimits_{#1}} 
\providecommand{\norm}[1]{\left\lVert#1\right\rVert}
%\providecommand{\norm}[1]{\lVert#1\rVert}
\providecommand{\mtx}[1]{\mathbf{#1}}
\providecommand{\mean}[1]{E\left[ #1 \right]}
\providecommand{\fourier}{\overset{\mathcal{F}}{ \rightleftharpoons}}
%\providecommand{\hilbert}{\overset{\mathcal{H}}{ \rightleftharpoons}}
\providecommand{\system}{\overset{\mathcal{H}}{ \longleftrightarrow}}
	%\newcommand{\solution}[2]{\textbf{Solution:}{#1}}
\newcommand{\solution}{\noindent \textbf{Solution: }}
\newcommand{\cosec}{\,\text{cosec}\,}
\providecommand{\dec}[2]{\ensuremath{\overset{#1}{\underset{#2}{\gtrless}}}}
\newcommand{\myvec}[1]{\ensuremath{\begin{pmatrix}#1\end{pmatrix}}}
\newcommand{\mydet}[1]{\ensuremath{\begin{vmatrix}#1\end{vmatrix}}}
\numberwithin{equation}{subsection}
\makeatletter
\@addtoreset{figure}{problem}
\makeatother
\let\StandardTheFigure\thefigure
\let\vec\mathbf
\renewcommand{\thefigure}{\theproblem}
\def\putbox#1#2#3{\makebox[0in][l]{\makebox[#1][l]{}\raisebox{\baselineskip}[0in][0in]{\raisebox{#2}[0in][0in]{#3}}}}
     \def\rightbox#1{\makebox[0in][r]{#1}}
     \def\centbox#1{\makebox[0in]{#1}}
     \def\topbox#1{\raisebox{-\baselineskip}[0in][0in]{#1}}
     \def\midbox#1{\raisebox{-0.5\baselineskip}[0in][0in]{#1}}
\vspace{3cm}
\title{Assignment 7}
\author{R.OOHA}
\maketitle
\newpage
\bigskip
\renewcommand{\thefigure}{\theenumi}
\renewcommand{\thetable}{\theenumi}
Download all python codes from 
\begin{lstlisting}p
https://github.com/ooharapolu/ASSIGNMENT7/Assignment7.py
\end{lstlisting}
%
and latex-tikz codes from 
%
\begin{lstlisting}
https://github.com/ooharapolu/ASSIGNMENT 7/main.tex
\end{lstlisting}
%
\section{QUESTION No-2.73 (Quadratic forms)}
 In each of the exercises,find the coordinates of the foci, the vertices,the length of major axis, the minor axis, the eccentricity and the length of the latus rectum of the ellipse 
 \\
 \vec{f}. $\vec{x}^{\top}\myvec{49 & 0 \\ 0 & -16}\vec{x} = 784$.
%
\section{Solution}
Given equation of the ellipse,
\begin{align}
  \vec{x}^{\top}\myvec{49 & 0 \\ 0 & -16}\vec{x} = 784
  \\
  \implies \vec{x}^{\top}\myvec{\frac{1}{16} & 0 \\ 0 & \frac{-1}{49}} = 1
\end{align}
we have,
\begin{align}
    \vec{V} = \myvec{\frac{1}{16} & 0 \\ 0 & \frac{-1}{49}}
    \\
    \vec{u}^{\top}\vec{V}^{-1}\vec{u}-f = 1
    \\
    \vec{c} = -\vec{V}^{-1}\vec{u}=\myvec{0 \\ 0}
    \\
    \lambda_1 = \frac{1}{16} , \lambda_2 = \frac{-1}{49}
\end{align}
Axes of ellipse is given by:
Length of semi major axis, a is
\begin{align}
  a = \sqrt{\frac{\vec{u}^{\top}\vec{V}^{-1}\vec{u}-f}{\lambda_1}} \label{eq:1}
 \end{align}
 substituting the values in \eqref{eq:1},we get
 \begin{align}
    a = 4
 \end{align}
Length of major axis is 2a = 8
\\
 and the length of semi minor axis, b is
  \begin{align}
    b = \sqrt{\frac{f-\vec{u}^{\top}\vec{V}^{-1}\vec{u}}{\lambda_2}} \label{eq:2}
 \end{align}
 substituting the values in \eqref{eq:2},we get
 \begin{align}
    b = 7 
\end{align}
 Length of the minor axis is 2b = 14
\\
The vertices are given as
\begin{align}
    \pm\myvec{4 \\ 0} 
\end{align}
Coordinates of foci are given by,
\begin{align}
  \vec{F} =\pm\brak{\sqrt{\frac{(\vec{u}^T\vec{V}^{-1}\vec{u}-f)(\lambda_2-\lambda_1)}{\lambda_1\lambda_2}}}\vec{p_1} \label{eq:3}
\end{align}
where, $\vec{p_1} = \myvec{1 \\ 0}$ since the equation of ellipse is in standard form.
Substituting the values in \eqref{eq:3} we have,
\begin{align}
    \vec{F} = \pm\myvec{\sqrt{65} \\ 0}.
\end{align}
Eccentricity of the ellipse is given by,
\begin{align}
   e = \frac{\sqrt{\frac{(\vec{u}^{\top}\vec{V}^{-1}\vec{u})(\lambda_2-\lambda_1)}{\lambda_1\lambda_2}}}{\sqrt{\frac{\vec{u}^{\top}\vec{V}^{-1}\vec{u}-f}{\lambda_1}}} \label{eq:4}
\end{align}
substituting the values in \eqref{eq:4},we have
\begin{align}
   e = \frac{\sqrt{65}}{4}.
\end{align}
Length of the latus rectum is given by,
\begin{align}
    l = \frac{2\brak{{\sqrt{\frac{f-\vec{u}^{\top}\vec{V}^{-1}\vec{u}}{\lambda_2}}}}^2}{\sqrt{\frac{\vec{u}^{\top}\vec{V}^{-1}\vec{u}-f}{\lambda_1}}} \label{eq:5}
\end{align}
substituting the values in \eqref{eq:5},we have
\begin{align}
   l = \frac{29}{2}
\end{align}
The plot of the ellipse is given below
\numberwithin{figure}{section}
\begin{figure}[ht]
\centering
\includegraphics[width=\columnwidth]{ellipse 7.png}
\caption{Plot of standard ellipse}
\label{Plot of standard ellipse}
\end{figure}
\end{document}